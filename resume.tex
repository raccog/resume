\documentclass{resume} % Use the custom resume.cls style

\usepackage[left=0.4 in,top=0.4in,right=0.4 in,bottom=0.4in]{geometry} % Document margins
\newcommand{\tab}[1]{\hspace{.2667\textwidth}\rlap{#1}}
\newcommand{\itab}[1]{\hspace{0em}\rlap{#1}}
\name{Ryan Cohen} % Your name
% You can merge both of these into a single line, if you do not have a website.
\address{+1(781) 724-9973 \\ Scituate, MA}
\address{\href{mailto:rcohenprogramming@gmail.com}{rcohenprogramming@gmail.com} \\ \href{https://github.com/raccog}{https://github.com/raccog}}  %

\begin{document}

%----------------------------------------------------------------------------------------
%	WORK EXPERIENCE SECTION
%----------------------------------------------------------------------------------------

\begin{rSection}{PERSONAL PROJECTS}
	\itemsep -8pt{}
	\item \href{https://github.com/raccog/caliga-bootloader}{\textbf{Caliga Bootloader}}\quad A boot loader project where I learned how to get information from and interact with modern hardware. Also learned how to interact with firmware such as UEFI and create a physical memory manager.
	\item \href{https://github.com/raccog/PumpkinComputer}{\textbf{Pumpkin Computer}}\quad Working on a RISC-V RTL design along with connected peripherals such as a JTAG module for debugging. My plan is to use this computer as an open-source hardware testbed for developing operating systems.
	\item \href{https://github.com/raccog/RyanOS}{\textbf{RyanOS}}\quad An operating system project with a custom UEFI boot loader. I learned how to run my operating system on emulators and on real hardware. Worked on a libc implementation and a UEFI interface library.
	\item \href{https://github.com/raccog/toy-bootboot}{\textbf{toy-bootboot}}\quad A project to re-implement a boot loader protocol on my own. Learned how to adhere to a protocol and how to create and run a UEFI boot loader.
	\item \href{https://github.com/raccog/chip8-rp2040}{\textbf{chip8-rp2040}}\quad An emulator for the CHIP-8 interpreted language on a microcontroller with an OLED screen.
	\item \href{https://github.com/raccog/ssd1306-rp2040}{\textbf{ssd1306-rp2040}}\quad A library for interfacing with a 128x64 OLED screen using I2C. Used in chip8-rp2040.
	\item \href{https://github.com/raccog/rraytracer}{\textbf{rraytracer}}\quad A ray tracer implementation based on the book "Ray Tracing in One Weekend" by Peter Shirley. Learned how a 2D image can be generated from a scene of 3D models using linear algebra.
	\item \href{https://github.com/raccog/rtga}{\textbf{rtga}}\quad A C library for interfacing with a TGA image file. Used in rraytracer for generating ray-traced images.
\end{rSection}

\begin{rSection}{OPEN SOURCE CONTRIBUTIONS}
	
	\vspace{-1.25em}
	\item
	\href{https://github.com/mortbopet/Ripes}{\textbf{Ripes}}
	(\href{https://github.com/mortbopet/Ripes/commits/master/?author=raccog}{My Commits})
	\vspace{-.75em}
	\begin{itemize}
		\itemsep -8pt{}
		\item Assisted in developing a generalized ISA assembler.
		\item Developed a library for defining generalized ISA information and instruction details at compile-time.
		\item Currently working on dynamic and filterable Qt interface for users to interact with a database of generalized ISA instructions.
		\href{https://github.com/mortbopet/Ripes/issues/297}{(Github Issue)}
		\href{https://github.com/mortbopet/Ripes/pull/329}{(Github Pull Request)}
		\item Fixed bugs and submitted bug reports.
	\end{itemize}
	\vspace{-.75em}
	\item  \href{https://git.savannah.gnu.org/cgit/grub.git}{\textbf{GRUB Boot Loader}}
	\vspace{-.75em}
	\begin{itemize}
		\itemsep -8pt{}
		\item Fixed two integer underflow vulnerabilities in the command line that affected all supported systems.
		\href{https://git.savannah.gnu.org/gitweb/?p=grub.git;a=commit;h=77afd25f8065bfbf5cc7848855006cd5260aeb9f}{(Commit)}
		\item Fixed an out-of-bounds write vulnerability in the VGA text module affecting some BIOS systems.
		\href{https://git.savannah.gnu.org/gitweb/?p=grub.git;a=commit;h=108a3865f43330b581d35b9cf6ecb1e0a1da5d49}{(Commit)}
	\end{itemize}
	\vspace{-.75em}
	\item \href{https://github.com/limine-bootloader/limine}{\textbf{Limine Boot Loader}}
	\vspace{-.75em}
	\begin{itemize}
		\itemsep -8pt{}
		\item Fixed a bug. \href{https://github.com/limine-bootloader/limine/commit/07d8dd2c68aea2bd4a03e8b69a521c67abc1d618}{(Commit)}
	\end{itemize}
	\vspace{-.75em}
	\item \href{https://github.com/rust-osdev/uefi-rs}{\textbf{UEFI-rs}}
	\vspace{-.75em}
	\begin{itemize}
		\itemsep -8pt{}
	
		\itemsep -8pt{}	\item Contributed documentation to understand better how the library's APIs work and how they could be used.
		\item Fixed a bug where the Rust panic handler function would log an unnecessary file name. \href{https://github.com/rust-osdev/uefi-rs/commit/5939e57f6786e60e9caf57f5aa0aa53b8bc3d83b}{(Commit)}
	\end{itemize}
	
\end{rSection}
%----------------------------------------------------------------------------------------


%----------------------------------------------------------------------------------------
% TECHINICAL STRENGTHS
%----------------------------------------------------------------------------------------
\vspace{-.75em}
\begin{rSection}{SKILLS}
\begin{description}
	\itemsep -8pt{}
	\item[Programming Languages] C/C++, Rust, Python, Bash, Assembly/Disassembly (x86\_64, ARM, RISC-V),\\
	Verilog/SystemVerilog
	\item[Development Tools] Git, Github Actions, GNU Make, CMake, GCC, QEMU, GDB, Ghidra, Vivado
	\item[Hardware Programming] x86\_64 (no OS), Microcontrollers, Xilinx FPGA, OLED Screens
	\item[Firmware] UEFI
	\item[Frameworks] Qt
\end{description}
\end{rSection}

%----------------------------------------------------------------------------------------
%	EDUCATION SECTION
%----------------------------------------------------------------------------------------

\begin{rSection}{EXPERIENCE}
	
\textbf{Mentee - Ripes} \hfill September 2023- \\
Linux Foundation \hfill \textit{Remote}
\vspace{-.75em}
\begin{itemize}
	\itemsep -8pt{}
	\item Worked with my two mentors and another intern to make the contributions described above, in the Open Source Contributions section.
	\item Discussed projects, bugs, features, and ideas with my mentors.
\end{itemize}

\vspace{-.75em}
\textbf{Assistant of IT} \hfill Summer 2018 \\
South Shore Charter Public School \hfill \textit{Norwell, MA}
\vspace{-.75em}
 \begin{itemize}
 	\itemsep -8pt{}
 	\item Set up 100+ new Chromebooks for students
     \item Helped teachers with IT problems
     \item Helped set up devices and networking in the new High School building
     \item Ensured that all Ethernet sockets worked properly on the newly built floor of the High School building
     \item Helped teachers set up technology in their classrooms (PCs, projectors, speakers, etc.)
 \end{itemize}

\textbf{Assistant of IT} \hfill Summer 2016 \\
South Shore Charter Public School \hfill \textit{Norwell, MA}
 \begin{itemize}
 	\itemsep -8pt{}
 	\item Set up 200+ new devices for students and teachers (iPads, Macbooks, PCs, and Chromebooks)
     \item Helped teachers with IT problems
     \item Helped teachers set up technology in their classrooms (PCs, projectors, speakers, etc.)
 \end{itemize}

\end{rSection}

\begin{rSection}{Education}

{\bf UMass Lowell}, Bachelor of Information Technology - In Progress, GPA: 3.94 \hfill {Spring 2023-} \\
{\bf Bristol Community College},  \hfill {2020-2022} \\
Associate of Computer Information Systems - Computer Science Transfer to UMass, GPA: 3.95 \\
{\bf Scituate High School}, High School Degree, GPA: 3.39 \hfill {2015-2019} \\

\end{rSection}


\end{document}
